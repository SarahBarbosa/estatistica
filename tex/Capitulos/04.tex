\chapter{A Distribuição das Funções de Variáveis Aleatórias}

\section{Funções de Variáveis Aleatórias}

Os experimentos são tipicamente projetados para direcionar uma ou mais quantidades que podem ser medidas efetivamente com o equipamento disponível. Um exemplo é a medição de partículas de alta energia com um contador \textit{Geiger}, que registra a incidência de partículas que alcançam o detector. Com base no conhecimento dos parâmetros do equipamento e das características da fonte radioativa, podemos modelar o número de eventos detectados por um intervalo de tempo como uma distribuição de Poisson, cuja média inicialmente é desconhecida. O contador Geiger, no entanto, não discrimina exclusivamente as partículas oriundas da fonte radioativa; ele também capta outras partículas, o chamado ``ruído de fundo''. Para quantificar apenas o ruído de fundo, podemos realizar um experimento controlado removendo a fonte e medindo as contagens que são exclusivamente ambientais. Isso nos fornece duas variáveis aleatórias distintas: uma para o total de contagens (fonte mais fundo) e outra apenas para o ruído de fundo, cada uma seguindo sua própria distribuição de probabilidade.

A inferência sobre a taxa de partículas originadas exclusivamente pela fonte depende das variáveis medidas. Em experimentos ideais, a distribuição probabilística da variável de interesse pode ser derivada das distribuições das variáveis medidas, facilitando uma estimativa mais precisa dos parâmetros da distribuição. Por exemplo, enquanto a diferença entre duas variáveis Poisson não segue uma distribuição Poisson no experimento com o contador Geiger, a diferença entre duas variáveis normalmente distribuídas retém a propriedade de normalidade.

O estudo da distribuição das funções de variáveis aleatórias é um tópico complexo que é coberto exaustivamente em livros sobre teoria da probabilidade, como \citet{ross2019introduction}. Este capítulo aborda tópicos e métodos selecionados que são aplicáveis a situações típicas encontradas pelo analista de dados. Há também casos em que não é possível ou prático buscar a distribuição de probabilidade de uma variável aleatória de interesse. Nesses casos, ainda é possível seguir métodos aproximados para estudar sua média e variância. Alguns desses são introduzidos neste capítulo e depois desenvolvidos completamente no capítulo seguinte.

\section{Combinação Linear de Variáveis Aleatórias}

As variáveis experimentais muitas vezes estão relacionadas por uma relação linear simples. Antes de estudar a distribuição de funções mais complexas de variáveis aleatórias, é útil entender o comportamento da combinação linear de variáveis. A combinação linear de $N$ variáveis aleatórias $X_i$ é uma variável $Y$ definida por
\begin{equation}
 Y = \sum_{i=1}^{N} a_i X_i,
\end{equation}
onde $a_i$ são coeficientes constantes. Um exemplo típico é o sinal detectado por um instrumento, que pode ser pensado como a soma do sinal intrínseco da fonte mais o ruído de fundo. As distribuições dos sinais de fundo e da fonte influenciarão as propriedades do sinal total detectado, e, portanto, é importante entender as propriedades estatísticas dessa relação para caracterizar o sinal da fonte.

