\chapter*{Resolução dos Problemas}
\addcontentsline{toc}{chapter}{Resolução dos Problemas}

\section*{Capítulo 1}
\addcontentsline{toc}{section}{Capítulo 1}

\begin{enumerate}[label=\textbf{1.\arabic*.}]
\item \label{r1.1} Considere o lançamento de quatro moedas iguais em sequência, onde $\text{Cara}$ é representada por $\text{C}$ e $\text{Coroa}$ por $\text{K}$. O espaço amostral para esse experimento é:
\begin{align*}
	\Omega &= \{\text{CCCC}, \text{CCCK}, \text{CCKC}, \text{CKCC}, \text{KCCC}, \text{CCKK}, \text{CKCK}, \text{KCKC}, \\ 
	&\qquad\text{CKKC}, \text{KKCC}, \text{KCCK}, \text{KKKC}, \text{KKCK}, \text{KCKK}, \text{CKKK}, \text{KKKK}\}.
\end{align*}
Assumindo que as moedas são imparciais, cada evento tem a mesma probabilidade de $\sfrac{1}{16}$. A probabilidade de obter duas caras e duas coroas, ou seja, os eventos $\{ \text{CCKK}, \text{CKCK}, \text{KCKC}, \text{CKKC}, \text{KKCC}, \text{KCCK}\}$ é $\sfrac{6}{16} = \sfrac{3}{8}$.

\item Considere os eventos:
\begin{align*}
	A = \{ \text{primeiro lançamento mostra número ímpar} \}, \quad \text{e} \quad
	B = \{ \text{soma dos dois dados é 9} \}.
\end{align*}
O evento $A$ ocorre nos resultados $1, 3,$ e $5$ (como cada evento tem uma probabilidade $\sfrac{1}{6}$), assim $P(A) = \sfrac{1}{2}$. Para calcular $P(B)$, os pares que somam 9 são $(3,6), (4,5), (5,4),$ e $(6,3)$, portanto $P(B) = \sfrac{4}{36} = \sfrac{1}{9}$. A probabilidade da união de $A$ e $B$ deve excluir a interseção deles para evitar duplicações, que são os lançamentos $(3,6)$ e $(5,4)$, onde $P(A \cap B) = \sfrac{1}{8}$. Portanto, considerando o terceiro axioma de Kolmogorov generalizado (\autoref{1.5}), temos que a probabilidade do primeiro lançamento mostra um número ímpar ou a soma dos dois dados é 9 é
\begin{equation*}
P(A \cup B) = P(A) + P(B) - P(A \cap B) = \dfrac{1}{2} + \dfrac{1}{9} - \dfrac{1}{18} = \dfrac{5}{18}.
\end{equation*}

\item Considere os eventos:
\begin{align*}
	A = \{ \text{obter número par} \}, \quad \text{e} \quad
	B = \{ \text{obter número maior que 4} \}.
\end{align*}
O evento $A$ ocorre nos resultados $2, 4,$ e $6$, assim $P(A) = \sfrac{3}{6}  = \sfrac{1}{2}$. Para calcular $P(B)$, os resultados maiores que 4 são $5$ e $6$, portanto $P(B) = \sfrac{2}{6} = \sfrac{1}{3}$. A interseção de $A$ e $B$ ocorre apenas em $6$, assim $P(A \cap B) = \sfrac{1}{6}$. Assim, pelo terceiro axioma de Kolmogorov generalizado (\autoref{1.5}), temos:
\begin{equation*}
P(A \cup B) = P(A) + P(B) - P(A \cap B) = \dfrac{1}{2} + \dfrac{1}{3} - \dfrac{1}{6} = \dfrac{2}{3}.
\end{equation*}

\item Considere os eventos:
\begin{align*}
	A = \{ \text{soma dos dois lançamentos é 8} \}, \quad \text{e} \quad
	B = \{ \text{o primeiro lançamento mostra 5} \}.
\end{align*}
O evento $A$ ocorre para as seguintes combinações de resultados: $(2,6), (3,5), (4,4), (5,3), (6,2)$ e portanto $P(A) = \sfrac{5}{36}$. O evento $B$ tem uma probabilidade $P(B) = \sfrac{1}{6}$. O evento $A \cap B$ tem probabilidade $P(A \cap B) = \sfrac{1}{36}$ (situação em que ambos os eventos ocorrem, no caso, quando ocorrer o resultado $(5,3)$). Como
\begin{equation*}
P(A) \cdot P(B) = \dfrac{5}{36} \cdot \dfrac{1}{6} = \dfrac{5}{1296} \neq P(A \cap B) = \dfrac{1}{36},
\end{equation*}
os dois eventos não são estatisticamente independentes.

\item Considere os seguintes eventos:
\begin{align*}
	A = \{ \text{o primeiro lançamento é par} \}, \quad \text{e} \quad
	B = \{ \text{o segundo lançamento é par} \}.
\end{align*}
O evento $A$ ocorre para $2,4,6$ (bem como o evento $B$) e portanto $P(A) = \sfrac{3}{6} = \sfrac{1}{2}$ (analogamente, $P(B) = \sfrac{1}{2}$). O evento $A \cap B$, que ocorre para as seguintes combinações de resultados: $(2,2), (2,4), (2,6), (4,2), (4,4), (4,6), (6,6), (6,2), (6,4)$, tem probabilidade $P(A \cap B) = \sfrac{9}{36} = \sfrac{1}{4}$. Como
\begin{equation*}
	P(A) \cdot P(B) = \dfrac{1}{2} \cdot \dfrac{1}{2} = \dfrac{1}{4} = P(A \cap B) = \dfrac{1}{4},
\end{equation*}
os dois eventos são estatisticamente independentes.

\item A probabilidade de retirar uma bola vermelha é $p = \sfrac{3}{5}$, já que 3 das 5 bolas são vermelhas. (a) Os sorteios consecutivos são independentes e, portanto, a probabilidade de duas bolas vermelhas consecutivas é $p = \sfrac{3}{5} \cdot \sfrac{3}{5} = \sfrac{9}{25}$, já que os sorteios são feitos com reposição. (b) A condição de que o primeiro sorteio foi uma bola vermelha é irrelevante para o resultado do segundo sorteio, e portanto a probabilidade é apenas $p = \sfrac{3}{5}$. A solução também pode ser obtida usando a definição de probabilidade condicional (\autoref{1.6}) e a definição de independência estatística (\autoref{1.8}),
\begin{equation*}
P(B|A) = \dfrac{P(A \cap B)}{P(A)} = \dfrac{P(A)\cdot P(B)}{P(A)} = P(B) = \dfrac{3}{5}
\end{equation*}
onde $A$ é o evento de uma bola vermelha no primeiro sorteio e $B$ é o evento de uma bola vermelha no segundo sorteio.

\item A probabilidade do dado der $1$ (evento $A$) é $P(A) = \sfrac{1}{6}$, e a probabilidade da soma ser $5$ (evento $B$) é $P(B) = \sfrac{4}{36}$, já que as sequências que resultam em uma soma de $5$ são apenas quatro: $(2,3), (3,2), (4,1), (1,4)$. (a) A probabilidade de $A$ dado $B$ é,
\begin{equation*}
P(A|B) = \dfrac{P(A \cap B)}{P(B)} = \dfrac{\sfrac{1}{36}}{\sfrac{4}{36}} = \dfrac{1}{4},
\end{equation*}
já que $P(A \cap B) = \sfrac{1}{36}$ representa a probabilidade da única combinação $(1, 4)$ ocorrer. (b) A probabilidade de $B$ dado $A$ é,
\begin{equation*}
	P(B|A) = \dfrac{P(B \cap A)}{P(A)} = \dfrac{\sfrac{1}{36}}{\sfrac{1}{6}} = \dfrac{1}{6},
\end{equation*}
(c) A probabilidade do primeiro lançamento ser $1$ e a soma ser $5$ é simplesmente,
\begin{equation*}
P(A \cap B) = P(A|B)\cdot P(B) = \dfrac{1}{4} \times \dfrac{4}{36} = \dfrac{1}{36}.
\end{equation*}
(d) Usando a \autoref{1.11}, podemos mostrar que,
\begin{equation*}
P(A | B) = \dfrac{P(B | A)P(A)}{P(B)} = \dfrac{\sfrac{1}{6} \times \sfrac{1}{6}}{\sfrac{4}{36}} = \dfrac{1}{4}.
\end{equation*}
o teorema de Bayes é verificado.

\item (a) Uma sequência específica tem uma probabilidade de $P(A) = (\sfrac{1}{2})^4 = \sfrac{1}{16}$, já que os quatro lançamentos são independentes, e cada um tem uma probabilidade de \sfrac{1}{2} de cara. (b) Precisamos contar o número de sequências que têm 2 moedas caindo cara (evento $B$). Usando o resultado do Problema \ref{r1.1}, vemos que existem 6 dessas sequências, e portanto a probabilidade é $P(B) = \sfrac{6}{16} = \sfrac{3}{8}$ por contagem direta. O evento $B$ inclui $A$, portanto a probabilidade da interseção é $P(A \cap B) = \sfrac{1}{16}$. A probabilidade de uma sequência cara-coroa-cara-coroa dado que duas moedas caíram com a face para cima é
\begin{equation*}
P(A|B) = \dfrac{P(A \cap B)}{P(B)} = \dfrac{\sfrac{1}{16}}{\sfrac{3}{8}} = \dfrac{1}{6}
\end{equation*}
(c) Para isso, podemos usar o teorema de Bayes (\autoref{1.11}), 
\begin{equation*}
P(B | A) = \dfrac{P(A | B)P(B)}{P(A)} = \dfrac{\sfrac{1}{6}\times\sfrac{3}{8}}{\sfrac{1}{16}} = 1. 
\end{equation*}
O conhecimento da sequência cara-coroa-cara-coroa de fato implica a certeza de que 2 moedas caíram cara.
\end{enumerate}


\section*{Capítulo 2}
\addcontentsline{toc}{section}{Capítulo 2}

\begin{enumerate}[label=\textbf{2.\arabic*.}]
	\item Para mostrar que a distribuição exponencial é devidamente normalizada, façamos:
	\begin{equation*}
	\int_0^{\infty} f(x)\, dx = \lambda \int_0^{\infty} e^{-\lambda x}\,dx = \lambda \left(-\dfrac{1}{\lambda}e^{-\lambda x}\right)_0^{\infty} = 1.
	\end{equation*}
	A média $\mu$ pode ser calculada usando a \autoref{2.7}:
	\begin{equation*}
		\mu = \int_{0}^{\infty} x f(x)\, dx = \lambda \int_{0}^{\infty} x e^{-\lambda x}\,dx = \lambda\left(-\dfrac{1}{\lambda}xe^{-\lambda x}\bigg|_{0}^{\infty} + \dfrac{1}{\lambda}\int_{0}^{\infty}e^{-\lambda x}\,dx\right) = \dfrac{1}{\lambda}
	\end{equation*}
	A variância $\sigma^2$ pode ser calculada usando a \autoref{2.12}:
	\begin{equation*}
	\sigma^2 = \mathbb{E}[x^2] - \mu^2 = \lambda\int_{0}^{\infty} x^2 e^{-\lambda x}\, dx - \left(\dfrac{1}{\lambda}\right)^2 = \lambda\left(-\dfrac{1}{\lambda}x^2e^{-\lambda x}\bigg|_{0}^{\infty} + \dfrac{2}{\lambda} \int_{0}^{\infty} xe^{-\lambda x}\,dx\right) - \left(\dfrac{1}{\lambda}\right)^2 = \dfrac{1}{\lambda^2}.
	\end{equation*}
	
	\item A variância da média amostral $\overline{x}$ pode ser calculada da seguinte maneira:
	\begin{equation*}
	\text{Var}(\overline{x}) = \text{Var}\left(\dfrac{1}{N} \sum_{i=1}^{N} x_i\right).
	\end{equation*}
	Usando a propriedade fornecida na \autoref{2.13}, ou seja, $\text{Var}(aX) = a^2 \cdot \text{Var}(X)$, temos,
	\begin{equation*}
		\text{Var}(\overline{x}) = \left(\dfrac{1}{N}\right)^2\text{Var}\left(\sum_{i=1}^{N} x_i\right).
	\end{equation*}
	Considerando o fato de que $\text{Var}(x + y) = \text{Var}(x) +  \text{Var}(y)$, , podemos reescrever a expressão anterior como:
	\begin{equation*}
		\text{Var}(\overline{x}) = \left(\dfrac{1}{N}\right)^2\left(\sum_{i=1}^{N} \text{Var}(x_i)\right).
	\end{equation*}
	Dado que $\text{Var}(x_i) = \sigma^2$ para $i = 1,2,\ldots,N$, obtemos que,
	\begin{equation*}
		\text{Var}(\overline{x}) = \left(\dfrac{1}{N}\right)^2\left(\sum_{i=1}^{N} \sigma^2\right) = \left(\dfrac{1}{N}\right)^2 \left(N\sigma^2\right) = \dfrac{\sigma^2}{N}.
	\end{equation*}
\end{enumerate}