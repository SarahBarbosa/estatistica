\chapter*{Resolução dos Problemas}
\addcontentsline{toc}{chapter}{Resolução dos Problemas}

\section*{Capítulo 1}
\addcontentsline{toc}{section}{Capítulo 1}

\begin{enumerate}[label=\textbf{1.\arabic*.}]
\item \label{r1.1} Considere o lançamento de quatro moedas iguais em sequência, onde $\text{Cara}$ é representada por $\text{C}$ e $\text{Coroa}$ por $\text{K}$. O espaço amostral para esse experimento é:
\begin{align*}
	\Omega &= \{\text{CCCC}, \text{CCCK}, \text{CCKC}, \text{CKCC}, \text{KCCC}, \text{CCKK}, \text{CKCK}, \text{KCKC}, \\ 
	&\qquad\text{CKKC}, \text{KKCC}, \text{KCCK}, \text{KKKC}, \text{KKCK}, \text{KCKK}, \text{CKKK}, \text{KKKK}\}.
\end{align*}
Assumindo que as moedas são imparciais, cada evento tem a mesma probabilidade de $\sfrac{1}{16}$. A probabilidade de obter duas caras e duas coroas, ou seja, os eventos $\{ \text{CCKK}, \text{CKCK}, \text{KCKC}, \text{CKKC}, \text{KKCC}, \text{KCCK}\}$ é $\sfrac{6}{16} = \sfrac{3}{8}$.

\item Considere os eventos:
\begin{align*}
	A = \{ \text{primeiro lançamento mostra número ímpar} \}, \quad \text{e} \quad
	B = \{ \text{soma dos dois dados é 9} \}.
\end{align*}
O evento $A$ ocorre nos resultados $1, 3,$ e $5$ (como cada evento tem uma probabilidade $\sfrac{1}{6}$), assim $P(A) = \sfrac{1}{2}$. Para calcular $P(B)$, os pares que somam 9 são $(3,6), (4,5), (5,4),$ e $(6,3)$, portanto $P(B) = \sfrac{4}{36} = \sfrac{1}{9}$. A probabilidade da união de $A$ e $B$ deve excluir a interseção deles para evitar duplicações, que são os lançamentos $(3,6)$ e $(5,4)$, onde $P(A \cap B) = \sfrac{1}{8}$. Portanto, considerando o terceiro axioma de Kolmogorov generalizado (\autoref{1.5}), temos que a probabilidade do primeiro lançamento mostra um número ímpar ou a soma dos dois dados é 9 é
\begin{equation*}
P(A \cup B) = P(A) + P(B) - P(A \cap B) = \dfrac{1}{2} + \dfrac{1}{9} - \dfrac{1}{18} = \dfrac{5}{18}.
\end{equation*}

\item Considere os eventos:
\begin{align*}
	A = \{ \text{obter número par} \}, \quad \text{e} \quad
	B = \{ \text{obter número maior que 4} \}.
\end{align*}
O evento $A$ ocorre nos resultados $2, 4,$ e $6$, assim $P(A) = \sfrac{3}{6}  = \sfrac{1}{2}$. Para calcular $P(B)$, os resultados maiores que 4 são $5$ e $6$, portanto $P(B) = \sfrac{2}{6} = \sfrac{1}{3}$. A interseção de $A$ e $B$ ocorre apenas em $6$, assim $P(A \cap B) = \sfrac{1}{6}$. Assim, pelo terceiro axioma de Kolmogorov generalizado (\autoref{1.5}), temos:
\begin{equation*}
P(A \cup B) = P(A) + P(B) - P(A \cap B) = \dfrac{1}{2} + \dfrac{1}{3} - \dfrac{1}{6} = \dfrac{2}{3}.
\end{equation*}

\item Considere os eventos:
\begin{align*}
	A = \{ \text{soma dos dois lançamentos é 8} \}, \quad \text{e} \quad
	B = \{ \text{o primeiro lançamento mostra 5} \}.
\end{align*}
O evento $A$ ocorre para as seguintes combinações de resultados: $(2,6), (3,5), (4,4), (5,3), (6,2)$ e portanto $P(A) = \sfrac{5}{36}$. O evento $B$ tem uma probabilidade $P(B) = \sfrac{1}{6}$. O evento $A \cap B$ tem probabilidade $P(A \cap B) = \sfrac{1}{36}$ (situação em que ambos os eventos ocorrem, no caso, quando ocorrer o resultado $(5,3)$). Como
\begin{equation*}
P(A) \cdot P(B) = \dfrac{5}{36} \cdot \dfrac{1}{6} = \dfrac{5}{1296} \neq P(A \cap B) = \dfrac{1}{36},
\end{equation*}
os dois eventos não são estatisticamente independentes.

\item Considere os seguintes eventos:
\begin{align*}
	A = \{ \text{o primeiro lançamento é par} \}, \quad \text{e} \quad
	B = \{ \text{o segundo lançamento é par} \}.
\end{align*}
O evento $A$ ocorre para $2,4,6$ (bem como o evento $B$) e portanto $P(A) = \sfrac{3}{6} = \sfrac{1}{2}$ (analogamente, $P(B) = \sfrac{1}{2}$). O evento $A \cap B$, que ocorre para as seguintes combinações de resultados: $(2,2), (2,4), (2,6), (4,2), (4,4), (4,6), (6,6), (6,2), (6,4)$, tem probabilidade $P(A \cap B) = \sfrac{9}{36} = \sfrac{1}{4}$. Como
\begin{equation*}
	P(A) \cdot P(B) = \dfrac{1}{2} \cdot \dfrac{1}{2} = \dfrac{1}{4} = P(A \cap B) = \dfrac{1}{4},
\end{equation*}
os dois eventos são estatisticamente independentes.

\item A probabilidade de retirar uma bola vermelha é $p = \sfrac{3}{5}$, já que 3 das 5 bolas são vermelhas. (a) Os sorteios consecutivos são independentes e, portanto, a probabilidade de duas bolas vermelhas consecutivas é $p = \sfrac{3}{5} \cdot \sfrac{3}{5} = \sfrac{9}{25}$, já que os sorteios são feitos com reposição. (b) A condição de que o primeiro sorteio foi uma bola vermelha é irrelevante para o resultado do segundo sorteio, e portanto a probabilidade é apenas $p = \sfrac{3}{5}$. A solução também pode ser obtida usando a definição de probabilidade condicional (\autoref{1.6}) e a definição de independência estatística (\autoref{1.8}),
\begin{equation*}
P(B|A) = \dfrac{P(A \cap B)}{P(A)} = \dfrac{P(A)\cdot P(B)}{P(A)} = P(B) = \dfrac{3}{5}
\end{equation*}
onde $A$ é o evento de uma bola vermelha no primeiro sorteio e $B$ é o evento de uma bola vermelha no segundo sorteio.

\item A probabilidade do dado der $1$ (evento $A$) é $P(A) = \sfrac{1}{6}$, e a probabilidade da soma ser $5$ (evento $B$) é $P(B) = \sfrac{4}{36}$, já que as sequências que resultam em uma soma de $5$ são apenas quatro: $(2,3), (3,2), (4,1), (1,4)$. (a) A probabilidade de $A$ dado $B$ é,
\begin{equation*}
P(A|B) = \dfrac{P(A \cap B)}{P(B)} = \dfrac{\sfrac{1}{36}}{\sfrac{4}{36}} = \dfrac{1}{4},
\end{equation*}
já que $P(A \cap B) = \sfrac{1}{36}$ representa a probabilidade da única combinação $(1, 4)$ ocorrer. (b) A probabilidade de $B$ dado $A$ é,
\begin{equation*}
	P(B|A) = \dfrac{P(B \cap A)}{P(A)} = \dfrac{\sfrac{1}{36}}{\sfrac{1}{6}} = \dfrac{1}{6},
\end{equation*}
(c) A probabilidade do primeiro lançamento ser $1$ e a soma ser $5$ é simplesmente,
\begin{equation*}
P(A \cap B) = P(A|B)\cdot P(B) = \dfrac{1}{4} \times \dfrac{4}{36} = \dfrac{1}{36}.
\end{equation*}
(d) Usando a \autoref{1.11}, podemos mostrar que,
\begin{equation*}
P(A | B) = \dfrac{P(B | A)P(A)}{P(B)} = \dfrac{\sfrac{1}{6} \times \sfrac{1}{6}}{\sfrac{4}{36}} = \dfrac{1}{4}.
\end{equation*}
o teorema de Bayes é verificado.

\item (a) Uma sequência específica tem uma probabilidade de $P(A) = (\sfrac{1}{2})^4 = \sfrac{1}{16}$, já que os quatro lançamentos são independentes, e cada um tem uma probabilidade de \sfrac{1}{2} de cara. (b) Precisamos contar o número de sequências que têm 2 moedas caindo cara (evento $B$). Usando o resultado do Problema \ref{r1.1}, vemos que existem 6 dessas sequências, e portanto a probabilidade é $P(B) = \sfrac{6}{16} = \sfrac{3}{8}$ por contagem direta. O evento $B$ inclui $A$, portanto a probabilidade da interseção é $P(A \cap B) = \sfrac{1}{16}$. A probabilidade de uma sequência cara-coroa-cara-coroa dado que duas moedas caíram com a face para cima é
\begin{equation*}
P(A|B) = \dfrac{P(A \cap B)}{P(B)} = \dfrac{\sfrac{1}{16}}{\sfrac{3}{8}} = \dfrac{1}{6}
\end{equation*}
(c) Para isso, podemos usar o teorema de Bayes (\autoref{1.11}), 
\begin{equation*}
P(B | A) = \dfrac{P(A | B)P(B)}{P(A)} = \dfrac{\sfrac{1}{6}\times\sfrac{3}{8}}{\sfrac{1}{16}} = 1. 
\end{equation*}
O conhecimento da sequência cara-coroa-cara-coroa de fato implica a certeza de que 2 moedas caíram cara.
\end{enumerate}


\section*{Capítulo 2}
\addcontentsline{toc}{section}{Capítulo 2}

\begin{enumerate}[label=\textbf{2.\arabic*.}]
	\item Para mostrar que a distribuição exponencial é devidamente normalizada, façamos:
	\begin{equation*}
	\int_0^{\infty} f(x)\, dx = \lambda \int_0^{\infty} e^{-\lambda x}\,dx = \lambda \left(-\dfrac{1}{\lambda}e^{-\lambda x}\right)_0^{\infty} = 1.
	\end{equation*}
	A média $\mu$ pode ser calculada usando a \autoref{2.7}:
	\begin{equation*}
		\mu = \int_{0}^{\infty} x f(x)\, dx = \lambda \int_{0}^{\infty} x e^{-\lambda x}\,dx = \lambda\left(-\dfrac{1}{\lambda}xe^{-\lambda x}\bigg|_{0}^{\infty} + \dfrac{1}{\lambda}\int_{0}^{\infty}e^{-\lambda x}\,dx\right) = \dfrac{1}{\lambda}
	\end{equation*}
	A variância $\sigma^2$ pode ser calculada usando a \autoref{2.12}:
	\begin{equation*}
	\sigma^2 = \mathbb{E}[x^2] - \mu^2 = \lambda\int_{0}^{\infty} x^2 e^{-\lambda x}\, dx - \left(\dfrac{1}{\lambda}\right)^2 = \lambda\left(-\dfrac{1}{\lambda}x^2e^{-\lambda x}\bigg|_{0}^{\infty} + \dfrac{2}{\lambda} \int_{0}^{\infty} xe^{-\lambda x}\,dx\right) - \left(\dfrac{1}{\lambda}\right)^2 = \dfrac{1}{\lambda^2}.
	\end{equation*}
	
	\item A variância da média amostral $\overline{x}$ pode ser calculada da seguinte maneira:
	\begin{equation*}
	\text{Var}(\overline{x}) = \text{Var}\left(\dfrac{1}{N} \sum_{i=1}^{N} x_i\right).
	\end{equation*}
	Usando a propriedade fornecida na \autoref{2.13}, ou seja, $\text{Var}(aX) = a^2 \cdot \text{Var}(X)$, temos,
	\begin{equation*}
		\text{Var}(\overline{x}) = \left(\dfrac{1}{N}\right)^2\text{Var}\left(\sum_{i=1}^{N} x_i\right).
	\end{equation*}
	Considerando o fato de que $\text{Var}(x + y) = \text{Var}(x) +  \text{Var}(y)$, , podemos reescrever a expressão anterior como:
	\begin{equation*}
		\text{Var}(\overline{x}) = \left(\dfrac{1}{N}\right)^2\left(\sum_{i=1}^{N} \text{Var}(x_i)\right).
	\end{equation*}
	Dado que $\text{Var}(x_i) = \sigma^2$ para $i = 1,2,\ldots,N$, obtemos que,
	\begin{equation*}
		\text{Var}(\overline{x}) = \left(\dfrac{1}{N}\right)^2\left(\sum_{i=1}^{N} \sigma^2\right) = \left(\dfrac{1}{N}\right)^2 \left(N\sigma^2\right) = \dfrac{\sigma^2}{N}.
	\end{equation*}
	
	\item Utilizando os conjuntos de dados para o Tubo 1 e o Tubo 2 (disponíveis em \mintinline{python}|../data/thomson1.dat| e \mintinline{python}|../data/thomson2.dat|, respectivamente), podemos calcular a média amostral (\autoref{2.8}), variância (\autoref{2.14}), covariância (\autoref{2.19}) e coeficiente de correlação (\autoref{2.22}). As fórmulas correspondentes foram implementadas em Python e podem ser encontradas em \mintinline{python}|../utils/justdoit.py|. Os resultados obtidos após a aplicação dessas fórmulas são os seguintes:
	
	\begin{itemize}
		\item Tubo 1:
		\begin{itemize}
			\item Média de $W/Q$: 13.3
			\item Variância de $W/Q$: 71.5
			\item Média de $I$: 312.9
			\item Variância de $I$: 8715.7
			\item Covariância entre $W/Q$ e $I$: 759.1
			\item Coeficiente de correlação entre $W/Q$ e $I$: 0.96
		\end{itemize}
		
		\item Tubo 2:
		\begin{itemize}
			\item Média de $W/Q$: 2.92
			\item Variância de $W/Q$: 0.67
			\item Média de $I$: 174.3
			\item Variância de $I$: 286.2
			\item Covariância entre $W/Q$ e $I$: 13.2
			\item Coeficiente de correlação entre $W/Q$ e $I$: 0.95
		\end{itemize}
	\end{itemize}
	
	Cabe ressaltar que, por conveniência, cada medição de $W/Q$ está faltando um fator de $10^{11}$. A aplicação das fórmulas correspondentes pode ser vista no Jupyter Notebook \mintinline{python}|../docs/notebooks/cap02.ipynb|.
	
	\item Ao combinar todas as medições feitas no ar (14 medições), no hidrogênio (5) e no ácido carbônico (4), os resultados são os seguintes:
	
	\begin{itemize}
	\item Ar: $m/e = 0.47 \pm 0.09$ (o desvio padrão é 0.09);		
	\item Hidrogênio: $m/e = 0.47 \pm 0.08$;
	\item Ácido carbônico: $m/e = 0.43 \pm 0.07$.
	\end{itemize}	
	
	Todas as medições estão dentro do desvio padrão umas das outras, e a afirmação é portanto verificada. As medições estão em unidades de $10^{-7}$. A aplicação das fórmulas correspondentes pode ser vista no Jupyter Notebook \mintinline{python}|../docs/notebooks/cap02.ipynb|.
	
	\item A covariância amostral é calculada como $-0.33$ e o coeficiente de correlação amostral como $-0.14$. A aplicação das fórmulas correspondentes pode ser vista no Jupyter Notebook \mintinline{python}|../docs/notebooks/cap02.ipynb|. 
\end{enumerate}

\section*{Capítulo 3}
\addcontentsline{toc}{section}{Capítulo 3}

\begin{enumerate}[label=\textbf{3.\arabic*.}]
	
	\item Para calcular os momentos da distribuição Gaussiana, precisamos ter em mente as seguintes integrais:
	\begin{equation*}
	\int_{-\infty}^{\infty} e^{-ax^2}\, dx = \sqrt{\dfrac{\pi}{a}}; \qquad \int_{-\infty}^{\infty}x e^{-ax^2}\, dx = 0, \qquad \text{e} \qquad \int_{-\infty}^{\infty}x^2 e^{-ax^2}\, dx = \dfrac{\sqrt{\pi}}{2}\dfrac{1}{a^{3/2}}.
	\end{equation*}
	A média pode ser calculada como,
	\begin{equation*}
		\mathbb{E}[X] = \int_{-\infty}^{\infty} x f(x)\, dx = \dfrac{1}{\sqrt{2\pi \sigma^2}} \int_{-\infty}^{\infty} x \exp\left[-\dfrac{(x-\mu)^2}{2\sigma^2}\right]\, dx,
	\end{equation*}
	fazendo $y = x - \mu$,
	\begin{equation*}
		\mathbb{E}[X] = \dfrac{1}{\sqrt{2\pi \sigma^2}} \int_{-\infty}^{\infty} (y + \mu) \exp\left(-\dfrac{y^2}{2\sigma^2}\right)\, dy =  \dfrac{1}{\sqrt{2\pi \sigma^2}} \int_{-\infty}^{\infty} \mu \exp\left(-\dfrac{y^2}{2\sigma^2}\right)\, dy = \mu.
	\end{equation*}
	Para calcular a variância, precisamos do momento de segunda ordem,
	\begin{equation*}
		\mathbb{E}[X^2] = \int_{-\infty}^{\infty} x^2 f(x)\, dx = \dfrac{1}{\sqrt{2\pi \sigma^2}} \int_{-\infty}^{\infty} x^2 \exp\left[-\dfrac{(x-\mu)^2}{2\sigma^2}\right]\, dx.
	\end{equation*}
	Fazendo $y = x - \mu$ e expandindo os termos,
	\begin{align*}
		\mathbb{E}[X^2] &= \dfrac{1}{\sqrt{2\pi \sigma^2}} \int_{-\infty}^{\infty} (y + \mu)^2 \exp\left(-\dfrac{y^2}{2\sigma^2}\right)\, dy \\
		&= \dfrac{1}{\sqrt{2\pi \sigma^2}} \left[\int_{-\infty}^{\infty} y^2 \exp\left(-\dfrac{y^2}{2\sigma^2}\right)\, dy + 2\mu\int_{-\infty}^{\infty}  y \exp\left(-\dfrac{y^2}{2\sigma^2}\right)\, dy + \mu^2\int_{-\infty}^{\infty}  \exp\left(-\dfrac{y^2}{2\sigma^2}\right)\, dy\right] \\
		&= \dfrac{1}{\sqrt{2\pi \sigma^2}} \left[\dfrac{\sqrt{\pi}}{2}\left(2\sigma^2\right)^{3/2} + \mu^2 \sqrt{2\pi\sigma^2} \right] \\ 
		&= \sigma^2 + \mu^2.
	\end{align*}
	Portanto, a variância é:
	\begin{equation*}
		\text{Var}(X) = \mathbb{E}[X^2] - \mathbb{E}[X]^2 = \sigma^2 + \mu^2 - \mu^2 = \sigma^2.
	\end{equation*}
	Os momentos centrais ímpares de ordem superior são,
	\begin{equation*}
		\mathbb{E}[(X - \mu)^{2n + 1}] = \int_{-\infty}^{\infty} (x - \mu)^{2n + 1} \exp\left[-\dfrac{(x-\mu)^2}{2\sigma^2}\right]\, dx = \int_{-\infty}^{\infty} y^{2n + 1} \exp\left[-\dfrac{y^2}{2\sigma^2}\right]\, dy = 0,
	\end{equation*}	
	onde novamente consideramos $y = x - \mu$. A integral acima é sempre zero para qualquer $n \geq 0$ devido à combinação da paridade ímpar de $y^{2n+1}$ e a natureza par da função exponencial. O produto de uma função ímpar por uma função par resulta em uma função ímpar, cuja integral sobre um intervalo simétrico como $-\infty$ a $\infty$ se cancela, garantindo que a integral total seja zero.
	
	\item (a) Dado que as pontuações seguem uma distribuição normal com média $\mu = 100$ e desvio padrão $\sigma = 15$, queremos encontrar a probabilidade de que um Q.I. seja maior ou igual a 145. Primeiro precisamos converter essa pontuação de Q.I. para um z-score (\autoref{3.12}), 
	\begin{equation*}
	Z = \dfrac{X - \mu}{\sigma} = \dfrac{145 - 100}{15} = 3
	\end{equation*}
	 Utilizando a \autoref{3.13} ou a calculadora online \href{https://www.calculator.net/z-score-calculator.html?c1raw=145&c1mean=100&c1sd=15&calctype=zscore&x=Calculate}{Calculator.net}, encontramos a probabilidade de $Z$ ser menor que 3, e subtraímos esse valor de 1, já que queremos a probabilidade de $Z$ ser maior ou igual a 3:
	\begin{equation*}
	P(Z \geq 3) = 1 - P(Z < 3) = 1 - 0.99865 = 0.00135.
	\end{equation*} 
	Assim, 0.135\% é a probabilidade de exceder a média em $3\sigma$ na direção positiva. (b) Para a média da pontuação de Q.I. de uma amostra de 100 pessoas, queremos calcular a probabilidade de que esta média seja igual ou maior que 105. A média amostral ($\bar{x}$) também segue uma distribuição normal, mas com o desvio padrão reduzido pelo fator de raiz do tamanho da amostra (\autoref{2.25}):
	\begin{equation*}
	\sigma_{\overline{x}} = \dfrac{\sigma}{\sqrt{N}} = \dfrac{15}{\sqrt{100}} = 1.5.
	\end{equation*}
	Convertendo para a $Z$-score:
	\begin{equation*}
	Z = \dfrac{\overline{x} - \mu}{\sigma_{\overline{x}}} = \dfrac{105 - 100}{1.5} \approx 3.33
	\end{equation*}
	Usando a distribuição normal \autoref{3.13} novamente ou a calculadora online \href{https://www.calculator.net/z-score-calculator.html?c1raw=105&c1mean=100&c1sd=1.5&calctype=zscore&x=Calculate}{Calculator.net},
	\begin{equation*}
	P(Z \geq 3.33) = 1 - P(Z < 3.33) \approx 1 - 0.99957 = 0.00043
	\end{equation*}
	Portanto, a probabilidade de que a média da pontuação de Q.I. de uma amostra de 100 pessoas seja igual ou maior que 105 é aproximadamente 0.043\%.
	
	\item (a) Isso pode ser calculado usando a distribuição binomial. Cada tentativa (lançamento da moeda) tem apenas dois resultados possíveis (sucessos ou fracassos) com probabilidades iguais $p = q = 0.5$ (caso a moeda seja justa). Usando a fórmula binomial (\autoref{3.5}), calculamos,
	\begin{equation*}
	P_{10}(5) = \binom{10}{5} (0.5)^5 (0.5)^{10 - 5} = \binom{10}{5}(0.5)^{10} \approx 0.246. 
	\end{equation*}
	(b) Nesse caso específico, cada sequência de jogadas é independente, então a probabilidade de um evento específico em que as primeiras 5 são caras e as últimas 5 são coroas é simplesmente:
	\begin{equation*}
	P(\text{5 caras seguidas e depois 5 coroas}) = (0.5)^5 \cdot (0.5)^5 = 0.000977.
	\end{equation*}
	(c) Para obter pelo menos 7 caras, somamos as probabilidades de obter 7, 8, 9 e 10 caras:
	\begin{align*}
	P(\text{pelo menos 7 caras}) &= P_{10}(7) + P_{10}(8) + P_{10}(9) + P_{10}(10) \\
	& = \binom{10}{7} (0.5)^{10} + \binom{10}{8} (0.5)^{10} + \binom{10}{9} (0.5)^{10} + \binom{10}{10} (0.5)^{10} \\
	&\approx 0.172.
	\end{align*}
	
	\item (a) O valor esperado em uma distribuição binomial é dado pela \autoref{3.6a}. Usando essa expressão, obtemos
	\begin{equation*}
	\mathbb{E}[X] = \mu = pN = (0.073)(32) = 2.336.
	\end{equation*}
	Portanto, o valor esperado de reprovações em uma turma de 32 alunos é aproximadamente 2.336 alunos. (b)	Para calcular a probabilidade de que 5 ou mais alunos reprovem, usamos a \autoref{3.5} e subtraímos de 1 para obter a probabilidade do complemento. Ou seja, para calcular $P(n \geq 5)$, somamos as probabilidades de $n = 5, 6, \dots, 32$. No entanto, é mais prático calcular $1 - P(n < 5)$, que é $P(n = 0) + P(n = 1) + P(n = 2) + P(n = 3) + P(n = 4)$. Assim, 
	\begin{equation*}
	P(n \geq 5) = 1 - P(n < 5) = 1 - \sum_{n=0}^{4} \binom{32}{n} (0.073)^n (0.927)^{32-n} \approx 0.0803,
	\end{equation*}	
	A probabilidade de que 5 ou mais alunos reprovem em uma turma de 32 alunos é aproximadamente 8.03\%.
	
	\item (a) Usando a \autoref{3.5}, a probabilidade de que não haja gêmeos em 200 nascimentos é:
	\begin{equation*}
	P(n = 0) =  \binom{200}{0} (0.012)^0 (1 - 0.012)^{200} \approx 0.0894
	\end{equation*}
	(b) A distribuição de Poisson é usada como uma aproximação da binomial quando $N$ é grande e $p$ é pequeno, o que é o caso aqui. Usando a \autoref{3.15} com $\mu = pN$, a probabilidade de $n = 0$ gêmeos:
	\begin{equation*}
	P(n = 0) = \dfrac{\mu^n}{n!}e^{- \mu} = \dfrac{(0.012 \cdot 200)^0}{0!}e^{-(0.012 \cdot 200)} \approx 0.091.
	\end{equation*}
\end{enumerate}